\documentclass{report}
\usepackage[T1, T2A]{fontenc}% T2A for Cyrillic font encoding
\usepackage[english, russian]{babel}
\usepackage{listings}

\lstloadlanguages{C,C++,csh,Java}
% \usepackage{mathptmx}
\usepackage{array}

\author{Сафонов Артём}
\title{Статический анализ}
\date{26.04.2025}

\begin{document}
\tableofcontents

% === #1 ===
\chapter{Анализ типов}

\section{Что будет, если в нашу систему типов ввести тип \(Bool\)?}

Продублируем изначальные правила:

\begingroup
\begin{center}
\renewcommand{\arraystretch}{1.5}
\begin{tabular}{ m{6cm} m{6cm} }
    \(I \) & \([[I]] = int\) \\  
    \(E_1 == E_2 \) & \([[E_1]] == [[E_2]] \wedge [[E_1 \ op \ E_2]] = int\) \\ 
    \(E_1 op E_2 \) & \([[E_1]] == [[E_2]] == [[E_1 \ op \ E_2]] = int\) \\ 
    \(input\) & \([[input]] = int\) \\
    \(X = E\) & \([[X]] = [[E]]\) \\
    \(output \ E\) & \([[E]] = int\) \\
    \(if(E) \ \{S_1\}\) & \([[E]] = int\) \\
    \(if(E) \ \{S_1\} \ else \ \{S_2\}\) & \([[E]] = int\) \\
    \(while \ (E) \ \{S\}\) & \([[E]] = int\) \\
    \(f(X_1,...,X_n) \ \{...return \ E;\}\) & \([[f]] = ([[X_1]],...,[[X_n]]) \rightarrow [[E]]\) \\
    \((E) \ (E_1,...,E_n)\) & \([[E]] = ([[E_1]],...,[[E_n]]) \rightarrow [[(E)(E_1,...,E_n)]]\) \\
    \(\&E\) & \([[\&E]] = \&[[E]]\) \\
    \(alloc\) & \([[alloc]] = \&\alpha\) \\
    \(null\) & \([[null]] = \&\alpha\) \\
    \(^*E\) & \([[E]] = \&[[^*E]]\) \\
    \(^*X=E\) & \([[X]] = \&[[E]]\) \\
\end{tabular}
\end{center}
\endgroup

Тогда, перво-наперво введём булевый литерал в пару к \(I\) - целочисленном литералу:

\begin{center}
\(B \Rightarrow [[B]] = boolean\)
\end{center}

Понятно, что возможные значения - это \textit{True} или \textit{False}. Следовательно меняется тип выражений в инструкциях, а также у бинарных операторов - теперь стоило бы выделить логические операторы и арифметические операторы, но т.к. в TIP есть только два логических оператора то нет нужды выписывать какой-нибудь \textit{logOp}.

Ещё одно следствие - мы не знаем тип \textit{input} и \textit{output}. Выпишем изменившиеся правила:

\begingroup
\begin{center}
\renewcommand{\arraystretch}{1.5}
\begin{tabular}{ m{6cm} m{6cm} }
    \(I \) & \([[I]] = int\) \\  
    \(B \) & \([[B]]] = boolean\) \\
    \(E_1 \ > \  E_2 \) & \([[E_1]] == [[E_2]] = int \wedge [[E_1 \ > \ E_2]] = boolean\) \\ 
    \(E_1 \ == \  E_2 \) & \([[E_1]] == [[E_2]] == [[E_1 \ == \ E_2]] = boolean\) \\ 
    \(E_1 \ op \ E_2 \) & \([[E_1]] == [[E_2]] == [[E_1 \ op \ E_2]] = int\) \\ 
    \(input\) & \([[input]] = \alpha\) \\
    \(X = E\) & \([[X]] = [[E]]\) \\
    \(output \ E\) & \([[E]] = \alpha\) \\
    \(if(E) \ \{S_1\}\) & \([[E]] = boolean\) \\
    \(if(E) \ \{S_1\} \ else \ \{S_2\}\) & \([[E]] = boolean\) \\
    \(while \ (E) \ \{S\}\) & \([[E]] = boolean\) \\
    \(f(X_1,...,X_n) \ \{...return \ E;\}\) & \([[f]] = ([[X_1]],...,[[X_n]]) \rightarrow [[E]]\) \\
    \((E) \ (E_1,...,E_n)\) & \([[E]] = ([[E_1]],...,[[E_n]]) \rightarrow [[(E)(E_1,...,E_n)]]\) \\
    \(\&E\) & \([[\&E]] = \&[[E]]\) \\
    \(alloc\) & \([[alloc]] = \&\alpha\) \\
    \(null\) & \([[null]] = \&\alpha\) \\
    \(^*E\) & \([[E]] = \&[[^*E]]\) \\
    \(^*X=E\) & \([[X]] = \&[[E]]\) \\
\end{tabular}
\end{center}
\endgroup

\subsection{Будет ли анализ более полным?}

Учитывая, что теперь в инструкциях \textit{if} и \textit{while} условие может быть только типа \textit{Bool}, следовало бы что полнота анализа увеличилась, например можно было бы найти ошибки когда в этих инструкциях условие - это арифметическое выражение, однако семантика языка не совпадает с этим правилом (в обоих инструкциях можно вставить целочисленное значение как условие), поэтому полнота всё-таки упадёт.

\subsection{Будет ли анализ более точным?}

Точность не изменится. Soundness как была такая и осталась.

\section{Что будет, если в нашу систему типов ввести тип \(Array\)?}

По аналогии введём литера массива, а также пустой массив:

\begin{center}
    \(\{ \; \} \Rightarrow [[\{ \; \}]] = [\alpha]\) \\
    \(\{E_1,...,E_n\} \Rightarrow \ [[E_1]] == \ ... \  == [[E2]] \) \\
    \(\ E = \{E_1,...,E_n\} \Rightarrow \ [[E]] == [ \ [[E_1]] \ ] \)
\end{center}

Все элементы массива долнжы иметь один тип, а вообще-то то есть это либо \textit{int} либо \textit{unit} (пустой массив), либо также массив, обозначим тип массива как \([<typename>] \).

\subsection{Придумайте правила вывода для новых операторов}
Введём операцию взятия индекса:

\begin{center}
    \( E[E_1] \Rightarrow [[E]] == [\alpha] \ \wedge \ [[E_1]] == int \ \wedge \ [[ \ E[E_1]\  ]] = \alpha\) \\
    \( E[E_1] = E_2 \Rightarrow [[E]] == [\alpha] \ \wedge \ [[E_1]] == int \ \wedge \ [[ \ E[E_1] \ ]] == [[E_2]] \  \wedge \ E_2 = \alpha\) \\
\end{center}

Индексация происходит по числу, следовательно тип индекса - это \textit{int}, также тип присваемого значения должен соответсвовать типу элемента массива, говоря иначе типу выражения, которое возвращает операция взятия индекса.

\newpage
\subsection{Попробуйте протипизировать программу}

Используя добавленые правила протипизируем программму:

\begingroup
\begin{center}
\begin{lstlisting}
main() {
    var x,y,z,t;
    x = {2,4,8,16,32,64};   [[x]]=[ [[2]] ]=[ int ]
    
    y = x[x[3]];            [[3]]=int^x=[int]=>[[x[3]]]=int=>[[x[x[3]]]]=int
                                =>[[y]]=int
                                
    z = {{},x};             [[{}]]=[a]^[[x]]=[int]=>[[z]]= [[int]]
    
    t = z[1];               [[z]]=[[int]]^[[1]]=int=>[[t]]=[int]
    
    t[2] = y;               [[y]]=int^[[2]]=int^[[t]]=[int]=>[[t[2]]]=int
}
\end{lstlisting}
% \renewcommand{\arraystretch}{1.5}
% \begin{tabular}{ m{6cm} m{6cm} }
%     var x,y,z,t; & a \\
%     x = {2,4,8,16,32,64}; & a \\
%     y = x[x[3]]; & a \\
%     z = {{},x}; & a \\
%     t = z[1]; & a \\
%     t[2] = y; & a \\
% \end{tabular}
\end{center}
\endgroup

В результате получаем:

\begin{center}
    \( [[x]] = [\ int\ ] \) \\
    \(  [[y]] = int \) \\
    \(  [[z]] = [\ [\ int\ ]\ ] \) \\
    \(  [[t]] = int \) \\
\end{center}

\end{document}
